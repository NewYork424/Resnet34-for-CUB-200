\documentclass[a4paper,11pt]{article}

% ==================== 宏包引入 ====================
\usepackage[UTF8]{ctex}     % 中文支持
\usepackage{geometry}       % 页面布局
\usepackage{graphicx}       % 图片插入
\usepackage{booktabs}       % 专业三线表
\usepackage{float}          % 浮动体控制
\usepackage{hyperref}       % 超链接
\usepackage{amsmath}        % 数学公式
\usepackage{amssymb}        % 数学符号
\usepackage{caption}        % 图表标题设置
\usepackage{dirtree}        % 目录树展示
\usepackage{listings}       % 代码块展示
\usepackage{xcolor}         % 颜色支持
\usepackage{fancyhdr}       % 页眉页脚

% ==================== 页面设置 ====================
\geometry{left=2.5cm, right=2.5cm, top=2.5cm, bottom=2.5cm}
\pagestyle{fancy}
\fancyhf{}
\rhead{模式识别与机器学习大作业报告}
\lhead{鸟类细粒度分类}
\cfoot{\thepage}

% 代码高亮样式
\lstset{
    basicstyle=\footnotesize\ttfamily,
    breaklines=true,
    frame=single,
    numbers=left,
    numberstyle=\tiny\color{gray},
    keywordstyle=\color{blue},
    commentstyle=\color{green!50!black},
    stringstyle=\color{red},
    language=Python
}

% ==================== 文档主体 ====================
\title{\textbf{基于显著性引导与自监督对比学习的\\鸟类细粒度分类系统设计}}
\author{姓名:聂溢 \quad 学号:2023010998}
\date{\today}

\begin{document}

\maketitle

% 目录
\tableofcontents
\newpage

% ==================== 正文 ====================

\section{任务说明与实验设置}

\subsection{任务描述}
本次大作业旨在对 CUB-200-2011 鸟类数据集进行分类。
根据作业要求,数据集包含 200 个类别,每类约 60 张图像。
硬件环境配置:NVIDIA RTX 3090 GPU。
任务分为两部分:
\begin{enumerate}
    \item \textbf{传统模式识别}:基于官方提供的属性特征,选取 10 类进行分类。
    \item \textbf{深度学习}:基于原始 RGB 图像进行 200 类全量分类。要求模型从头训练,禁止使用外部预训练权重。
\end{enumerate}


\section{传统机器学习方法实验结果}

本节对比了支持向量机、决策树和线性模型在 10 类鸟类属性特征上的表现。

\begin{table}[H]
    \centering
    \caption{传统机器学习方法性能对比,基于属性特征}
    \label{tab:traditional_comparison}
    \begin{tabular}{lccc}
        \toprule
        \textbf{模型} & \textbf{特征类型} & \textbf{关键参数} & \textbf{准确率} \\
        \midrule
        SVM & Attribute & Kernel=RBF, C=10 & 0.9825 \\
        Linear Model & Attribute & Map=Poly2, LR=0.05 & 0.9825 \\
        Decision Tree & Attribute & Criterion=CART & 0.7544 \\
        \bottomrule
    \end{tabular}
\end{table}

支持向量机与线性模型均取得了 98.25\% 的极高准确率。
这表明官方提供的 384 维属性特征,如"是否有白色腹部",在高维空间中具有极佳的可分性。

\section{方法论与系统设计}

该数据集分类任务中数据量少、"类间差异微小"与"背景环境复杂",
且不使用 ImageNet 预训练权重的前提下,根据以下原则构建模型。

\subsection{总体设计思路}
\begin{enumerate}
    \item \textbf{关注差异点}:鸟类分类往往依赖于头部、翅膀纹理等细微特征。
        模型必须具备空间位置敏感性,以在特征图中"高亮"这些区域。
    \item \textbf{忽略噪声}:CUB-200 数据集中包含大量树叶、水面等复杂背景。
        模型需主动抑制非主体区域的激活值。
    \item \textbf{防止过拟合}:在仅有约 6000 张训练样本且从零训练的情况下,深度模型极易过拟合。
        需通过强先验知识引入和数据增强来提升泛化能力。
\end{enumerate}

\subsection{关注差异点}

\subsubsection{SE 注意力机制}
Squeeze-and-Excitation 注意力机制,其通过全局平均池化将空间特征压缩为通道描述符,随后通过两层全连接网络学习通道间的依赖关系:
\begin{equation}
    \mathbf{s} = \sigma(W_2 \cdot \text{ReLU}(W_1 \cdot \text{GAP}(\mathbf{X})))
\end{equation}
其中 $\mathbf{s}$ 是通道权重向量,$\sigma$ 为 Sigmoid 激活函数。

SE 模块计算量小,参数量少,约为原网络的 1\% 左右,能够有效提升模型对关键特征通道的敏感度,在通用分类任务中表现稳定。
但由于采用全局平均池化,SE 注意力完全丢失了空间位置信息。

\subsubsection{坐标注意力机制}
针对 SE 注意力丢失空间信息的问题,坐标注意力机制,Coordinate Attention,
将特征图分别沿水平 $X$ 和垂直 $Y$ 方向进行池化:
\begin{equation}
    z^h_c(h) = \frac{1}{W}\sum_{0 \leq i < W} x_c(h, i), \quad
    z^w_c(w) = \frac{1}{H}\sum_{0 \leq j < H} x_c(j, w)
\end{equation}
这两个一维特征向量分别编码了"第 $h$ 行"和"第 $w$ 列"的全局上下文。
随后通过共享的 $1 \times 1$ 卷积层进行特征融合,最终生成水平和垂直两个方向的注意力权重图:
\begin{equation}
    \mathbf{A}^h = \sigma(W^h \cdot f), \quad \mathbf{A}^w = \sigma(W^w \cdot f)
\end{equation}
最终输出为 $\mathbf{Y} = \mathbf{X} \odot \mathbf{A}^h \odot \mathbf{A}^w$。

坐标注意力在保留通道建模能力的同时,显式引入了行列位置信息。这使得网络能够捕捉长距离的空间依赖关系,在"哪个位置需要关注"这一问题上具有更强的表达能力。
对于细粒度分类任务,坐标注意力能够更精准地定位鸟类头部、尾部等关键区域,尤其在复杂背景下表现出更好的鲁棒性。

\subsubsection{两种注意力机制的对比总结}

后面的实验发现,坐标注意力并非在所有场景下都优于 SE。
在不加预训练、数据增强的前提下,SE 注意力由于参数量更少、过拟合风险更低,有时能取得更优的性能。
但在添加了较好的预防过拟合手段后,坐标注意力的空间建模能力逐渐显现出优势。

\subsubsection{广义平均池化}
在特征聚合阶段,我们采用广义平均池化,GeM Pooling,其中 $p=3.0$,替代标准平均池化:
\begin{equation}
    \mathbf{f} = \left( \frac{1}{|\mathcal{X}|} \sum_{x \in \mathcal{X}} x^p \right)^{\frac{1}{p}}
\end{equation}
当 $p > 1$ 时,池化过程更关注激活值较高的区域。这有助于保留特征图中响应最强烈的"显著点",从而在最终分类时突出主体特征。

\subsection{忽略噪声}
为了实现"忽略背景噪声"的设计目标,本文设计了一种无需额外标注的辅助监督信号 $\mathcal{L}_{sal}$。
假设图像中高频纹理区域,如羽毛,的局部方差显著高于平滑背景,如天空,我们首先基于图像局部方差生成伪显著性图 $M_{sal}$。
随后,在训练过程中计算特征图空间注意力 $A_{feat}$ 与 $M_{sal}$ 的均方误差:
\begin{equation}
    \mathcal{L}_{sal} = \text{MSE}(A_{feat}, M_{sal})
\end{equation}
当模型错误地关注到背景,如树枝,时,$\mathcal{L}_{sal}$ 会产生较大的惩罚梯度,迫使网络抑制背景区域的激活。
总损失函数定义为:$\mathcal{L}_{total} = \mathcal{L}_{ce} + \alpha \mathcal{L}_{sal}$,其中 $\alpha=0.15$。

\subsection{防止过拟合}
针对"从零训练易过拟合"的问题,使用基于 MoCo v2 的域内自监督预训练策略。
我利用训练集数据进行了对比学习预训练。
其原理是,构建查询编码器 $q$ 和动量键编码器 $k$,通过 InfoNCE Loss 最大化同一图像不同增强视图的相似度。
这一阶段为骨干网络提供了一个比随机初始化更鲁棒的参数起点,显著降低了后续监督训练陷入局部最优解的风险。


\section{深度学习方法实验结果}

本节详细分析了基于改进 ResNet-34 的 200 类全量分类表现,并通过消融实验验证各设计思路的有效性。

\subsection{主要结果与消融实验}
表 \ref{tab:ablation_study} 展示了不同配置下的模型在验证集上的最终准确率 。

\begin{table}[H]
    \centering
    \caption{深度学习模型消融实验结果汇总}
    \label{tab:ablation_study}
    \resizebox{\textwidth}{!}{
    \begin{tabular}{lccccc}
        \toprule
        \textbf{Exp ID} & \textbf{Attention} & \textbf{RandAugment} & \textbf{Saliency Loss} & \textbf{MoCo Pretrain} & \textbf{Accuracy} \\
        \midrule
        1 & Coord & True & True & True & 81.53\% \\
        2 & SE & True & False & True & 81.36\% \\
        3 & SE & True & True & True & 81.02\% \\
        4 & SE & False & False & False & 80.10\% \\
        5 & SE & False & False & True & 79.60\% \\
        6 & SE & False & True & True & 79.18\% \\
        7 & SE & True & True & False & 78.84\% \\
        8 & Coord & True & True & False & 78.25\% \\
        9 & SE & False & True & False & 72.63\% \\
        10 (Baseline) & SE & False & False & False & 69.69\% \\
        \bottomrule
    \end{tabular}
    }
\end{table}

\subsection{结果分析:验证设计思路}

\subsubsection{注意力机制的作用分析}
对比 Exp 1 与 Exp 8 可以发现,在相同配置下,坐标注意力,Coord,相比 SE 注意力提升了 3.28\%。
进一步对比 Exp 1 与 Exp 3,在引入强数据增强和 MoCo 预训练的情况下,坐标注意力相比 SE 仍有 0.51\% 的提升。

这一结果验证了我们在 3.2 节中的分析:坐标注意力的空间位置建模能力在细粒度分类任务中具有显著优势。
特别是在没有强先验知识,如 MoCo 预训练,的情况下,Exp 8 vs Baseline 提升 8.56\%,坐标注意力能够更有效地帮助网络定位关键区域。

然而,需要指出的是,Exp 2 的结果表明,当引入了足够强的正则化手段,如 RandAugment 和 MoCo,后,SE 注意力同样能够取得 81.36\% 的较好性能。
这说明注意力机制的效果受到大量因素的影响,并非坐标注意力在所有场景下都是最优解。

\subsubsection{去噪与聚焦能力的验证}
实验结果显示,在移除显著性损失后,即 Exp 2 与 Exp 1 对比,模型性能下降了 0.17\%;而在无强数据增强的基线模型上,显著性损失带来了近 3\% 的提升,即 Exp 9 与 Exp 10 对比。
因此,"噪声抑制"思路是有效的:当模型缺乏强正则化时,显式地告诉模型"哪里是背景"至关重要。

\subsubsection{抗过拟合策略的必要性}
对比实验中最显著的差异来自于 MoCo 预训练,对比 Exp 7 与 Exp 2 提升 2.5\%,和 RandAugment,对比 Exp 10 与 Exp 4 提升 10.4\%。
这充分说明,在小样本,每类仅 30 张图,且无 ImageNet 权重的情况下,单纯依靠网络结构改进是不够的。必须通过对比学习挖掘数据潜在信息,并利用强增强扩充数据边界,才能有效落实\textbf{"防止过拟合"}的设计目标。

\section{总结}
本次实验实现 CUB-200 鸟类分类任务。
\begin{itemize}
    \item \textbf{传统方法}:证明了属性特征的高线性可分性,准确率达 98.25\%。
    \item \textbf{深度学习}:实验结果有力地支撑了本文提出的设计思路,通过坐标注意力\textbf{聚焦差异},显著性损失\textbf{过滤噪声},以及 MoCo 与数据增强\textbf{对抗过拟合}。我们成功在零外部依赖的严苛条件下,将 ResNet-34 的准确率从基线 69.69\% 提升至 \textbf{81.53\%}。
    \item \textbf{注意力机制选择}:消融实验表明,坐标注意力在细粒度分类任务中具有优势,但在强正则化条件下,SE 注意力同样能取得接近的性能。注意力机制的选择应根据任务特点、数据规模和计算资源综合考虑。
\end{itemize}

\section{附录:项目结构}

\dirtree{%
.1 project/.
.2 config.yml\DTcomment{实验全局配置文件}.
.2 data/\DTcomment{数据集目录}.
.3 train/\DTcomment{训练集}.
.3 val/\DTcomment{验证集}.
.2 logs/\DTcomment{训练日志与模型权重}.
.2 report/\DTcomment{实验报告}.
.2 src/\DTcomment{源代码目录}.
.3 main.py\DTcomment{程序主入口}.
.3 decision\_tree\_model/\DTcomment{决策树模块}.
.4 decision\_tree.py\DTcomment{C4.5/CART实现}.
.4 grid\_search.py\DTcomment{网格搜索}.
.4 run\_tree.py\DTcomment{训练脚本}.
.3 deep\_learning/\DTcomment{深度学习模块}.
.4 resnet.py\DTcomment{ResNet模型定义}.
.4 contrastive\_pretrain.py\DTcomment{对比学习预训练}.
.4 run\_deeplearn.py\DTcomment{训练脚本}.
.3 linear\_model/\DTcomment{线性模型模块}.
.4 linear\_model.py\DTcomment{Softmax回归实现}.
.4 grid\_search.py\DTcomment{网格搜索}.
.4 run\_linear.py\DTcomment{训练脚本}.
.3 svm/\DTcomment{SVM模块}.
.4 run\_svm.py\DTcomment{SVM训练脚本}.
.4 grid\_search.py\DTcomment{网格搜索}.
.3 utils/\DTcomment{通用工具}.
.4 dataset.py\DTcomment{统一数据加载接口}.
.4 log.py\DTcomment{日志工具}.
}

\end{document}